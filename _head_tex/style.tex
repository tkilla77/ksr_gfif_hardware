% define my exercise
\newcounter{aufgabe} % start counter
\setcounter{aufgabe}{1}

\setlength{\parindent}{0pt} % offset on RHS (if 0, no indent)
\setlength{\parskip}{15pt} % distance btw exercise title and the exercise text, and next exercise title

\renewcommand{\d}{\mathrm{d}}
\newcommand{\mynewex}[1]{
{\bf Exercise \theaufgabe}
\stepcounter{aufgabe} #1 }

%\topmargin-10mm % decrease top margin (less distance btw top edge of sheet and txt)


%
% Define colors
\definecolor{myblue}{rgb}{0.1875,0.503906,0.929688}
\definecolor{myorange}{rgb}{1,0.54902,0}

\usepackage[
    margin=1.0in,
    bottom=1in,
    top=1.8cm,
    headsep=0.5cm
    ]{geometry}

\fancypagestyle{mypagestyle}{%
  \fancyhf{}% Clear header/footer
  \fancyhead[OL]{\class/\teachershort} %on Odd page, right
  \fancyhead[EL]{\class/\teachershort } %on Even page, left
  \fancyhead[OR]{\datum} % Title on Even page, right
  \fancyhead[ER]{\datum}% Title on Even page, Centred
  \fancyfoot[C]{\thepage}%
  \fancyfoot[R]{v: \ddmmyyyydate\today}%
  \renewcommand{\headrulewidth}{.4pt}% Header rule of .4pt
}
\pagestyle{mypagestyle}


\usepackage[framemethod=TikZ]{mdframed}
%\usepackage[framemethod=default]{mdframed}
% for auto-split frame environment, load after amsthm
%http://www.pirbot.com/mirrors/ctan/macros/latex/contrib/mdframed/mdframed.pdf

\newcounter{definition}[section]
\renewcommand{\thedefinition}{\thesection.\arabic{definition}}

\newcounter{theorem}[section]
\renewcommand{\thetheorem}{\thesection.\arabic{theorem}}

\newcounter{assignment}[section]
\renewcommand{\theassignment}{\thesection.\arabic{assignment}}

\newcounter{auftrag}[section]
\renewcommand{\theauftrag}{\thesection.\arabic{auftrag}}

\newcounter{example}[section]
\renewcommand{\theexample}{\thesection.\arabic{example}}

%\newenvironment{example}[1][]{%
%    \refstepcounter{example}
%    \textbf{Example~\theexample:\\#1}
%}


\newenvironment{definition}[1][]{%
    \refstepcounter{definition}
    \begin{mdframed}[%
        frametitle={Definition~\thedefinition\ #1},
        linecolor=myorange,
        skipabove=\baselineskip plus 2pt minus 1pt,
        skipbelow=\baselineskip plus 2pt minus 1pt,
        linewidth=0.8pt,
        splittopskip=30pt,
        splitbottomskip=30pt
%        frametitlerule=true
%        frametitlebackgroundcolor=gray!30
    ]%
}{%
    \end{mdframed}
}

\newenvironment{theorem}[1][]{%
    \refstepcounter{theorem}
    \begin{mdframed}[%
        frametitle={Theorem~\thetheorem\ #1},
        linecolor=green,
        skipabove=\baselineskip plus 2pt minus 1pt,
        skipbelow=\baselineskip plus 2pt minus 1pt,
        linewidth=0.8pt,
        splittopskip=30pt,
        splitbottomskip=30pt
%        frametitlerule=true
%        frametitlebackgroundcolor=gray!30
    ]%
}{%
    \end{mdframed}
}

% ENGLISH

\newenvironment{assignment}[1][]{%
    \refstepcounter{assignment}
    \begin{mdframed}[%
        frametitle={Task~\theassignment\ #1},
        linecolor=blue,
        skipabove=\baselineskip plus 2pt minus 1pt,
        skipbelow=\baselineskip plus 2pt minus 1pt,
        linewidth=0.8pt,
        splittopskip=30pt,
        splitbottomskip=30pt
%        frametitlerule=true
%        frametitlebackgroundcolor=gray!30
    ]%
}{%
    \end{mdframed}
}

%\newenvironment{example}[1][]{%
%    \refstepcounter{example}
%    \begin{mdframed}[%
%        frametitle={Example~\theexample\ #1},
%        linecolor=green,
%        skipabove=\baselineskip plus 2pt minus 1pt,
%        skipbelow=\baselineskip plus 2pt minus 1pt,
 %       linewidth=0.8pt,
 %       splittopskip=30pt,
 %       splitbottomskip=30pt
%%        frametitlerule=true
%%        frametitlebackgroundcolor=gray!30
%    ]%
%}{%
%    \end{mdframed}
%}

%% COLORED FRAMES WITHOUT LABEL & NUMBERING

\newenvironment{important}[1][]{%
    \begin{mdframed}[%
        linecolor=red,
        skipabove=\baselineskip plus 2pt minus 1pt,
        skipbelow=\baselineskip plus 2pt minus 1pt,
        linewidth=0.8pt
%        frametitlerule=true
%        frametitlebackgroundcolor=gray!30
    ]%
}{%
    \end{mdframed}
}

\newenvironment{boxred}[1][]{%
    \begin{mdframed}[%
        linecolor=red,
        skipabove=\baselineskip plus 2pt minus 1pt,
        skipbelow=\baselineskip plus 2pt minus 1pt,
        linewidth=0.8pt
%        frametitlerule=true
%        frametitlebackgroundcolor=gray!30
    ]%
}{%
    \end{mdframed}
}

\newenvironment{boxorange}[1][]{%
    \begin{mdframed}[%
        linecolor=myorange,
        skipabove=\baselineskip plus 2pt minus 1pt,
        skipbelow=\baselineskip plus 2pt minus 1pt,
        linewidth=0.8pt
%        frametitlerule=true
%        frametitlebackgroundcolor=gray!30
    ]%
}{%
    \end{mdframed}
}

\newenvironment{boxgreen}[1][]{%
    \begin{mdframed}[%
        linecolor=green,
        skipabove=\baselineskip plus 2pt minus 1pt,
        skipbelow=\baselineskip plus 2pt minus 1pt,
        linewidth=0.8pt
%        frametitlerule=true
%        frametitlebackgroundcolor=gray!30
    ]%
}{%
    \end{mdframed}
}

\newenvironment{boxblue}[1][]{%
    \begin{mdframed}[%
        linecolor=blue,
        skipabove=\baselineskip plus 2pt minus 1pt,
        skipbelow=\baselineskip plus 2pt minus 1pt,
        linewidth=0.8pt
%        frametitlerule=true
%        frametitlebackgroundcolor=gray!30
    ]%
}{%
    \end{mdframed}
}

%%%%%%%%%%%%%%%

% For various stuff, e.g., important calculation methods
\newenvironment{various}[1][]{%
    \begin{mdframed}[%
        linecolor=magenta,
        skipabove=\baselineskip plus 2pt minus 1pt,
        skipbelow=\baselineskip plus 2pt minus 1pt,
        linewidth=0.8pt
%        frametitlerule=true
%        frametitlebackgroundcolor=gray!30
    ]%
}{%
    \end{mdframed}
}


% DEUTSCH

\newenvironment{auftrag}[1][]{%
    \refstepcounter{auftrag}
    \begin{mdframed}[%
        frametitle={Auftrag~\theauftrag\ #1},
        linecolor=blue,
        skipabove=\baselineskip plus 2pt minus 1pt,
        skipbelow=\baselineskip plus 2pt minus 1pt,
        linewidth=0.8pt,
        splittopskip=30pt,
        splitbottomskip=30pt
%        frametitlerule=true
%        frametitlebackgroundcolor=gray!30
    ]%
}{%
    \end{mdframed}
}

%% HEADING BOX

\newcommand{\framedheading}[1]{%
	\begin{mdframed}[
		linecolor=black,
		skipabove=\baselineskip plus 2pt minus 1pt,
		skipbelow=\baselineskip plus 2pt minus 1pt,
		%		outerlinewidth=1pt,
		linewidth=0.8pt,
		%		splitbottomskip=30pt,
		%		splittopskip=30pt,
		%rightline=false,leftline=false
		backgroundcolor=black!5
		]
		{\large\bf\begin{center}#1\end{center}
		}
	\end{mdframed}
	\vspace{-0.5cm}
}


\newcommand{\framedsubheading}[1]{%
	\begin{mdframed}[
		linecolor=black,
		skipabove=\baselineskip plus 2pt minus 1pt,
		skipbelow=\baselineskip plus 2pt minus 1pt,
		%		outerlinewidth=1pt,
		linewidth=0.4pt,
		%		splitbottomskip=30pt,
		%		splittopskip=30pt,
		rightline=false,leftline=false
		%backgroundcolor=black!5
		]
		{\normalsize\bf\begin{center}#1\end{center}
		}
	\end{mdframed}
	\vspace{-0.5cm}
}


% TIKZ COMMANDS
\makeatletter
\newcommand{\gettikzxy}[3]{%
  \tikz@scan@one@point\pgfutil@firstofone#1\relax
  \edef#2{\the\pgf@x}%
  \edef#3{\the\pgf@y}%
}










